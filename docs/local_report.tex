\documentclass[12pt]{article}
\usepackage[utf8]{inputenc}
\usepackage[T1]{fontenc}
\usepackage[brazil]{babel}
\usepackage{graphicx}
\usepackage{booktabs}
\usepackage{geometry}
\usepackage{amsmath}
\graphicspath{{../results/}{../results/trial_figures/}}
\geometry{margin=2cm}

\title{Relatório Local de Detecção de ERD (OpenBMI)}
\author{Pipeline ERD Detector}
\date{\today}

\begin{document}
\maketitle

\section{Configuração do experimento}
\begin{itemize}
  \item Baseline: $[-2.0, -1.0]$ s (auto-referência C3/C4).
  \item Janela de detecção: $[0, 4]$ s.
  \item Limiar: $-0{,}60\,\sigma$ (1 canal com peso temporal).
  \item Peso temporal gaussiano: centro 0,6 s; $\sigma = 0{,}25$ s; piso 0,3.
  \item Artefatos: corte em 750 $\mu$V; IMF selecionadas $\geq 35\%$ de potência em 8--30 Hz.
  \item Passo temporal: 5 ms (0,005 s).
\end{itemize}

\section{Resultados Globais}
\begin{tabular}{ll}
\toprule
Total de tentativas & 400 \\
Detecções & 257 (63,2\%) \\
Ensaios limpos & 347 \\
Latência média & 519 ms \\
Latência mediana & 600 ms \\
Latência desvio-padrão & 109 ms \\
Latência mín--máx & 350 -- 800 ms \\
\bottomrule
\end{tabular}

\section{Figuras agregadas}
\subsection{Distribuição de latências}
\begin{center}
\includegraphics[width=0.9\linewidth]{local_latency_hist.png}
\end{center}
\subsection{Taxa de detecção por sujeito-sessão}
\begin{center}
\includegraphics[width=0.9\linewidth]{local_detection_bar.png}
\end{center}

\section{Fórmulas utilizadas}
\begin{itemize}
  \item Potência HHT: $P(t) = \left|\mathcal{H}\left(\text{IMF}_{\text{sel}}(t)\right)\right|^2$.
  \item Baseline: $\mu_b = \frac{1}{N}\sum_{t \in [-2,-1]} P(t)$; $\sigma_b = \sqrt{\frac{1}{N}\sum (P(t)-\mu_b)^2}$.
  \item Z-score: $z = \frac{\overline{P_{\text{janela}}} - \mu_b}{\sigma_b}$.
  \item Peso temporal: $w(t) = \max\left(0.3, \exp\left(-\tfrac{1}{2}\left(\frac{t-0.6}{0.25}\right)^2\right)\right)$.
  \item Critério: somar $w(t)$ para janelas com $z \leq -0.60\,\sigma$ até atingir 1 (equivalente a um canal “cheio”).
\end{itemize}

\section{Exemplos de detecção em ensaios reais}
Cada página seguinte mostra: EEG bruto e filtrado (C3/C4), IMF principal por canal, Z-score com baseline/limiar/peso gaussiano e potência HHT com baseline/limiar destacados (faixa cinza = baseline, faixa amarela = janela de detecção, linha preta = cue, linha verde = detecção).

\newpage\begin{center}\includegraphics[width=0.85\linewidth]{trial_subj01_sess01_idx003.png}\end{center}
\newpage\begin{center}\includegraphics[width=0.85\linewidth]{trial_subj01_sess01_idx005.png}\end{center}
\newpage\begin{center}\includegraphics[width=0.85\linewidth]{trial_subj01_sess01_idx010.png}\end{center}
\newpage\begin{center}\includegraphics[width=0.85\linewidth]{trial_subj01_sess01_idx014.png}\end{center}
\newpage\begin{center}\includegraphics[width=0.85\linewidth]{trial_subj01_sess01_idx016.png}\end{center}
\newpage\begin{center}\includegraphics[width=0.85\linewidth]{trial_subj01_sess01_idx020.png}\end{center}
\newpage\begin{center}\includegraphics[width=0.85\linewidth]{trial_subj01_sess01_idx021.png}\end{center}
\newpage\begin{center}\includegraphics[width=0.85\linewidth]{trial_subj01_sess01_idx033.png}\end{center}
\newpage\begin{center}\includegraphics[width=0.85\linewidth]{trial_subj01_sess01_idx034.png}\end{center}
\newpage\begin{center}\includegraphics[width=0.85\linewidth]{trial_subj01_sess01_idx043.png}\end{center}
\newpage\begin{center}\includegraphics[width=0.85\linewidth]{trial_subj01_sess01_idx044.png}\end{center}
\newpage\begin{center}\includegraphics[width=0.85\linewidth]{trial_subj01_sess01_idx049.png}\end{center}
\newpage\begin{center}\includegraphics[width=0.85\linewidth]{trial_subj01_sess01_idx051.png}\end{center}
\newpage\begin{center}\includegraphics[width=0.85\linewidth]{trial_subj01_sess01_idx052.png}\end{center}
\newpage\begin{center}\includegraphics[width=0.85\linewidth]{trial_subj01_sess01_idx054.png}\end{center}
\newpage\begin{center}\includegraphics[width=0.85\linewidth]{trial_subj01_sess01_idx060.png}\end{center}
\newpage\begin{center}\includegraphics[width=0.85\linewidth]{trial_subj01_sess01_idx061.png}\end{center}
\newpage\begin{center}\includegraphics[width=0.85\linewidth]{trial_subj01_sess01_idx064.png}\end{center}
\newpage\begin{center}\includegraphics[width=0.85\linewidth]{trial_subj02_sess01_idx034.png}\end{center}
\newpage\begin{center}\includegraphics[width=0.85\linewidth]{trial_subj02_sess01_idx070.png}\end{center}

\section{Interpretação rápida}
\begin{itemize}
  \item Processamento em lotes: apenas ~4 arquivos (8 GB) ficam no disco por vez; o CSV final soma 400 ensaios.
  \item Baseline longo e peso temporal gaussiano reduziram detecções precoces e estabilizaram o limiar.
  \item Artefatos marcados (>750 $\mu$V) permanecem registrados para inspeção manual.
\end{itemize}

\end{document}
